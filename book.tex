\documentclass[12pt, a4paper]{memoir}
\usepackage[top=25mm, bottom=25mm, left=20mm, right=20mm]{geometry}
\usepackage{hyperref}
\usepackage{minted} % For code highlighting (requires -shell-escape enabled in Overleaf)
\usepackage{graphicx}

\hypersetup{
    colorlinks=true,
    linkcolor=blue,
    urlcolor=cyan,
    pdftitle={School Maths Old and New},
    pdfauthor={Your Name}
}

% Title and author details
\title{Old School New School Maths}
\author{Your Name}
\date{\today}

\begin{document}

\frontmatter
\maketitle
\tableofcontents

\mainmatter

% Part 1: Modern Methodologies
\part{Modern Methodologies in Mathematics Teaching}

\chapter{Universal Design for Learning (UDL)}
\section{Introduction to UDL}
Incorporate UDL principles into your mathematics classroom to provide multiple means of engagement, representation, and expression. Explore interactive examples through tools like GeoGebra and Python-based simulations.

\section{Applying UDL in Algebra}
Here, we'll explore how to differentiate lessons on algebra using Python. Visit this GeoGebra activity: \href{https://www.geogebra.org/classic}{GeoGebra Link}.

\begin{minted}{python}
# Example Python code for generating algebra problems
import random

def generate_linear_equation():
    a = random.randint(1, 10)
    b = random.randint(1, 10)
    print(f"Solve for x: {a}x + {b} = 0")

generate_linear_equation()
\end{minted}

% Part 2: ICT Integration and Problem Solving
\part{ICT Integration and Problem Solving in Mathematics}

\chapter{Using Python for Problem Solving}
\section{Why Python?}
Python is an excellent tool for integrating coding into math lessons. This chapter covers examples of how Python can solve complex mathematical problems, enhance understanding, and provide visualizations.

\section{Python-Based Simulations}
Engage students with Python simulations that model real-world scenarios, such as graphing functions or simulating probability events.

\begin{minted}{python}
import matplotlib.pyplot as plt
import numpy as np

x = np.linspace(-10, 10, 100)
y = x**2

plt.plot(x, y)
plt.title("Graph of y = x^2")
plt.xlabel("x")
plt.ylabel("y")
plt.show()
\end{minted}

% Part 3: Problem Solving Approach
\part{Problem Solving Approach in Mathematics}

\chapter{Non-linear Approach to Learning Outcomes}
\section{Structuring Content Non-Linearly}
Instead of a linear approach based on learning outcomes, this chapter will focus on problem-solving strategies. Each problem links to various mathematical concepts, encouraging exploration.

\chapter{Interactive Problem Sets with GeoGebra}
Explore this interactive padlet: \href{https://padlet.com/yourpadletlink}{Problem Solving Padlet}. Use these resources alongside the Python code provided to solve complex equations interactively.

\backmatter
\chapter{References}
A collection of all resources, further reading materials, and external links such as \href{https://github.com/YourGithubRepo}{Your GitHub Repository} for the book.

\end{document}
