\documentclass[12pt, a4paper, addpoints]{exam}
\usepackage[ddmmyy, useregional]{datetime}
\usepackage{amsmath, array, booktabs, multicol, pgfplots, tabularx, pgfmath, xcolor, mdframed}

% Geometry settings for consistent margins
\usepackage[top=5mm, bottom=20mm, left=11mm, right=11mm]{geometry}
\pagestyle{empty}

% Document Metadata
\title{\large Simultaneous Equations using Elimination \\ \dayofweekname{\day}{\month}{\year} \today \\ \vspace{2mm} \quad Student's Name: \underline{\hspace{8cm}}}
\date{}

\newcommand{\elimination}[6]{%
    % Calculate constants c and p for the equations based on the intersection point (#1,#2)
    \pgfmathsetmacro{\cval}{#3*#1 + #5*#2} % c = a*x + b*y
    \pgfmathsetmacro{\pval}{#4*#1 + #6*#2} % p = d*x + e*y

    % Format terms for better readability, omitting "1" and aligning signs
    \def\formata{%
        \ifnum#3=1
            x
        \else\ifnum#3=-1
            -x
        \else
            \pgfmathprintnumber{#3}x
        \fi\fi}

    \def\formatb{%
        \ifnum#5=1
            +y
        \else\ifnum#5=-1
            -y
        \else
            \ifnum#5<0 \pgfmathprintnumber{#5}y \else +\pgfmathprintnumber{#5}y \fi
        \fi\fi}

    \def\formatd{%
        \ifnum#4=1
            x
        \else\ifnum#4=-1
            -x
        \else
            \pgfmathprintnumber{#4}x
        \fi\fi}

    \def\formate{%
        \ifnum#6=1
            +y
        \else\ifnum#6=-1
            -y
        \else
            \ifnum#6<0 \pgfmathprintnumber{#6}y \else +\pgfmathprintnumber{#6}y \fi
        \fi\fi}

    % Display equations without the left large bracket
    $\begin{aligned}
    \formata \formatb &= \pgfmathprintnumber{\cval} \\[-1pt]
    \formatd \formate &= \pgfmathprintnumber{\pval}
    \end{aligned}$
}

% Custom spacing
\newcommand{\as}{\vspace{0 mm}}
\newcommand{\bs}{\vspace{1 mm}}
\newcommand{\cs}{\vspace{2 mm}}
\newcommand{\ds}{\vspace{3 mm}}
\newcommand{\es}{\vspace{8 mm}}
\newcommand{\fs}{\vspace{10 mm}}
\newcommand{\gs}{\vspace{15 mm}}
\newcommand{\hs}{\vspace{20 mm}}
\newcommand{\is}{\vspace{25 mm}}
\newcommand{\js}{\vspace{30 mm}}
\newcommand{\ks}{\vspace{40 mm}}
\newcommand{\ls}{\vspace{50 mm}}
\newcommand{\ms}{\vspace{55 mm}}
\newcommand{\ns}{\vspace{60mm}}
\newcommand{\os}{\vspace{65 mm}}
\newcommand{\ps}{\vspace{75 mm}}
\newcommand{\qs}{\vspace{85mm}}
\newcommand{\rs}{\vspace{95 mm}}

% Begin Document
\begin{document}
\maketitle

\LARGE
\noindent 
In this unit, we will explore the elimination method for solving simultaneous equations. Check your answers and aim to understand each step fully.

% Instruction Frame
\begin{mdframed}[backgroundcolor=gray!10]
    \begin{multicols}{2}
        \large
        \begin{itemize}
            \item Align terms of both equations to eliminate variables.
            \item Add or subtract to eliminate one variable.
            \item Substitute to find the remaining variable.
            \item Verify solutions by substitution.
            \item Work with a partner for clarity and improvement.
        \end{itemize}
    \end{multicols}
\end{mdframed}

% Table of Contents
   \LARGE
% \begin{multicols}{2}
    \tableofcontents
% \end{multicols}
\newpage

% Questions Section
\begin{questions}
 
    % Adding sections for each question type, with file inputs
    \addcontentsline{toc}{section}{Opposite individual  \( y \) and \( -y \) terms}
    
\question Eliminate the opposite \(y\) and \(-y\) terms by addition.\hs
\begin{multicols}{2} % Start two-column layout
\begin{parts}

\part \elimination{2}{3}{3}{5}{1}{-1}\ms % Positive single-digit solution expected
\part \elimination{3}{2}{4}{6}{1}{-1}\ms % Positive single-digit solution expected
\part \elimination{1}{4}{2}{5}{1}{-1}\ms % Positive single-digit solution expected
\part \elimination{2}{5}{3}{4}{1}{-1}\ms % Positive single-digit solution expected
\part \elimination{1}{2}{5}{3}{1}{-1}\ms % Positive single-digit solution expected
\part \elimination{2}{1}{4}{6}{1}{-1}\ms % Positive single-digit solution expected

\end{parts}
\end{multicols}\newpage

  \addcontentsline{toc}{section}{Opposite   \( ay \) and \( -ay \) terms}
    \question Solve these pairs of simultaneous equations using elimination:\hs
\begin{multicols}{2}\ms % Start two-column layout
\begin{parts}
\part \elimination{2}{3}{3}{5}{4}{-4}\ms % Expected positive single-digit solution
\part \elimination{3}{2}{2}{6}{5}{-5}\ms % Example pair with positive single-digit solution
\part \elimination{3}{4}{5}{2}{3}{-3}\ms % Example pair
\part \elimination{2}{-1}{3}{1}{4}{-4}\ms % Example pair
\part \elimination{3}{-5}{5}{4}{2}{-2}\ms % Example pair
\part \elimination{7}{8}{3}{5}{4}{-4}\ms % Example pair

\end{parts}
\end{multicols}\newpage

    \addcontentsline{toc}{section}{Multiply One Line}
    \question Solve these pairs of simultaneous equations using elimination:\is
\begin{multicols}{2}\ms % Start two-column layout
\begin{parts}

\part \elimination{4}{3}{3}{2}{2}{-1}\ms % Solutions: x=4, y=3
\part \elimination{5}{2}{4}{3}{3}{-1}\ms % Solutions: x=5, y=2
\part \elimination{2}{4}{5}{3}{1}{-5}\ms % Solutions: x=2, y=4
\part \elimination{3}{1}{3}{1}{2}{-1}\ms % Solutions: x=3, y=1
\part \elimination{6}{2}{3}{2}{4}{-2}\ms % Solutions: x=6, y=2
\part \elimination{1}{5}{2}{3}{10}{-5}\ms % Solutions: x=1, y=5

\end{parts}
\end{multicols}\newpage
\end{questions}

\end{document}
