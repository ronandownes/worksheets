% Elimination worksheet

\documentclass[12pt, a4paper, addpoints]{exam}
\usepackage{amsmath}
\usepackage[margin=12mm]{geometry}
\usepackage{pgfmath}
\usepackage{xcolor}
\usepackage{multicol} % For multi-column layout
\usepackage{mdframed} % For framed boxes

% Define the new elimination command
\newcommand{\elimination}[6]{%
    % #1 = constant1, #2 = constant2, #3 = x-coeff1, #4 = x-coeff2, #5 = y-coeff1, #6 = y-coeff2
    % Compute constants c and p based on given coefficients and x, y values
    \pgfmathsetmacro{\cval}{#3*#1 + #5*#2} % c = a*constant1 + b*constant2
    \pgfmathsetmacro{\pval}{#4*#1 + #6*#2} % p = d*constant1 + e*constant2

    % Format coefficients (removing 1 or -1)
    \def\formata{%
        \ifnum#3=1
            x
        \else\ifnum#3=-1
            -x
        \else
            \pgfmathprintnumber{#3}x
        \fi\fi}

    \def\formatb{%
        \ifnum#5=1
            +y
        \else\ifnum#5=-1
            -y
        \else
            \ifnum#5<0 \pgfmathprintnumber{#5}y \else +\pgfmathprintnumber{#5}y \fi
        \fi\fi}

    \def\formatd{%
        \ifnum#4=1
            x
        \else\ifnum#4=-1
            -x
        \else
            \pgfmathprintnumber{#4}x
        \fi\fi}

    \def\formate{%
        \ifnum#6=1
            +y
        \else\ifnum#6=-1
            -y
        \else
            \ifnum#6<0 \pgfmathprintnumber{#6}y \else +\pgfmathprintnumber{#6}y \fi
        \fi\fi}

    % Display equations in aligned format without braces
    \[
    \begin{aligned}
    \formata \formatb &= \pgfmathprintnumber{\cval} \\[-1pt]
    \formatd \formate &= \pgfmathprintnumber{\pval}
    \end{aligned}
    \]
}

\begin{document}

\section*{Simultaneous Equations - Elimination by Addition}

\begin{mdframed}[backgroundcolor=gray!20, roundcorner=5pt] % Light grey background with rounded corners
\begin{multicols}{2}
\begin{enumerate}
    \item Multiply one or both equations if needed.
    \item Add or subtract to eliminate one variable.
    \item Solve for the remaining variable.
    \item Substitute back to find the second variable.
    \item Present solution as  \((x, y) = (2, 3)\) etc.
    \item Check solution in both original equations.
\end{enumerate}
\end{multicols}
\end{mdframed}

\newcommand{\verticalspace}{\vspace{21mm}} 

\large
% Define the question with two columns
\begin{questions}

% 1
\question Eliminate the opposite terms by addition
\begin{multicols}{2} % Start three-column layout
\begin{parts}

\part \elimination{2}{3}{3}{5}{1}{-1} \verticalspace
\part \elimination{3}{4}{4}{6}{1}{-1} \verticalspace
\part \elimination{4}{-5}{1}{-1}{4}{3} \verticalspace
\part \elimination{-2}{-3}{3}{4}{1}{-1} \verticalspace

\end{parts}
\end{multicols}

% 2
\question Eliminate opposite terms by addition
\begin{multicols}{2} % Start three-column layout
\begin{parts}

\part \elimination{2}{3}{3}{5}{4}{-4} \verticalspace
\part \elimination{4}{1}{2}{6}{5}{-5} \verticalspace
\part \elimination{7}{-2}{4}{2}{3}{-3} \verticalspace
\part \elimination{-1}{-2}{3}{1}{2}{-2} \verticalspace

\end{parts}
\end{multicols}
\newpage
\renewcommand{\verticalspace}{\vspace{35mm}} 
% 3
\question Multiply One Equation and continue
\begin{multicols}{2} % Start three-column layout
\begin{parts}

\part \elimination{4}{3}{3}{2}{2}{-1} \verticalspace
\part \elimination{5}{2}{4}{3}{3}{-1} \verticalspace
\part \elimination{2}{4}{5}{3}{1}{-1} \verticalspace
\part \elimination{3}{1}{3}{-6}{2}{5} \verticalspace

\end{parts}
\end{multicols}

% 4
\question Multiply both equations and continue
\begin{multicols}{2} % Start three-column layout
\begin{parts}

\part \elimination{4}{3}{2}{3}{3}{-5} \verticalspace
\part \elimination{5}{2}{3}{4}{2}{-3} \verticalspace
\part \elimination{2}{4}{4}{2}{7}{-5} \verticalspace
\part \elimination{3}{1}{2}{4}{5}{-3} \verticalspace

\end{parts}
\end{multicols}
\end{questions}

\end{document}
