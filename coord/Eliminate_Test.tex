% Elimination worksheet


\documentclass[12pt, a4paper, addpoints]{exam}
\usepackage{amsmath}
\usepackage[margin=12mm]{geometry}
\usepackage{pgfmath}
\usepackage{xcolor}
\usepackage{multicol} % For multi-column layout
\usepackage{mdframed} % For framed boxes
% Define a custom spacing command


\newcommand{\elimination}[6]{%
    % Compute constants c and p based on given coefficients and x, y values
    \pgfmathsetmacro{\cval}{#3*#1 + #4*#2} % c = a*x_value + b*y_value
    \pgfmathsetmacro{\pval}{#5*#1 + #6*#2} % p = d*x_value + e*y_value

    % Format coefficients (removing 1 or -1)
    \def\formata{%
        \ifnum#3=1
            x
        \else\ifnum#3=-1
            -x
        \else
            \pgfmathprintnumber{#3}x
        \fi\fi}

    \def\formatb{%
        \ifnum#4=1
            +y
        \else\ifnum#4=-1
            -y
        \else
            \ifnum#4<0 \pgfmathprintnumber{#4}y \else +\pgfmathprintnumber{#4}y \fi
        \fi\fi}

    \def\formatd{%
        \ifnum#5=1
            x
        \else\ifnum#5=-1
            -x
        \else
            \pgfmathprintnumber{#5}x
        \fi\fi}

    \def\formate{%
        \ifnum#6=1
            +y
        \else\ifnum#6=-1
            -y
        \else
            \ifnum#6<0 \pgfmathprintnumber{#6}y \else +\pgfmathprintnumber{#6}y \fi
        \fi\fi}

    % Display equations in aligned format without braces
    \[
    \begin{aligned}
    \formata \formatb &= \pgfmathprintnumber{\cval} \\[-1pt]
    \formatd \formate &= \pgfmathprintnumber{\pval}
    \end{aligned}
    \]
}
\begin{document}


\begin{minipage}{0.3\textwidth}
\section*{Elimination Test}
\end{minipage}
\begin{minipage}{0.7\textwidth}
\raggedleft
\textbf{Student's Name:} \underline{\hspace{7cm}} \\[0.2cm]
\textbf{Percentage:} \underline{\hspace{7cm}} \\
\end{minipage}
\vspace{11mm}
\begin{mdframed}[backgroundcolor=gray!20, roundcorner=5pt] % Light grey background with rounded corners
\textbf{Steps for Solving Using Elimination:}
\begin{multicols}{2}
\begin{enumerate}
    \item Multiply one or both equations if needed.
    \item Add or subtract to eliminate one variable.
    \item Solve for the remaining variable.
    \item Substitute back to find the second variable.
    \item Present solution as \((x, y) = (2, 3)\), etc.
    \item Check solution in both original equations.
\end{enumerate}
\end{multicols}
\end{mdframed}

\newcommand{\verticalspace}{\vspace{81mm}} 



\large
% Define the question with two columns
\begin{questions}
% 1
\question Use the elimination method to solve all four equations.
\begin{multicols}{2} 
\begin{parts}
\part \elimination{3}{2}{1}{3}{-1}{4} \columnbreak
\part \elimination{2}{1}{3}{6}{5}{-6}
\newpage
\part \elimination{6}{2}{3}{1}{2}{-3} \verticalspace
% \part \elimination{1}{5}{2}{1}{3}{-2} \verticalspace

% \end{parts}
% \end{multicols}

% \question Multiply both equations and continue
% \begin{multicols}{2} % Start three-column layout
% \begin{parts}

% \part \elimination{4}{3}{2}{3}{3}{-5} \verticalspace
% \part \elimination{5}{2}{3}{2}{4}{-3} \verticalspace
% \part \elimination{2}{4}{4}{7}{2}{-5} \verticalspace
% \part \elimination{3}{1}{2}{5}{4}{-3} \verticalspace
\part \elimination{6}{2}{3}{5}{4}{-3} \verticalspace
% \part \elimination{1}{5}{2}{3}{4}{-3} \verticalspace
\end{parts}
\end{multicols}
\end{questions}

\end{document}
