\documentclass[12pt, a4paper, addpoints]{exam}
\usepackage{tikz}
\usepackage{multicol}
\usepackage{mdframed}
\usepackage{geometry}

% Define the linesegment command
\newcommand{\linesegment}[6]{% Arguments: #1=grid size, #2=xstart, #3=ystart, #4=xend, #5=yend
\begin{tikzpicture}
    % Draw a grid with size specified by #1 x #2
    \draw[step=1cm, gray, very thin] (0,0) grid (#1,#2);
    
    % Plot the line segment from (xstart, ystart) to (xend, yend)
    \draw[black, thick] (#3,#4) -- (#5,#6);

    % Add endpoints
    \filldraw [black] (#3,#4) circle (2pt); % Endpoint at starting point
    \filldraw [black] (#5,#6) circle (2pt); % Endpoint at ending point
\end{tikzpicture}
}

\begin{document}

\begin{mdframed}[backgroundcolor=gray!20] % Light grey background
 \textbf{Slope} is the \textit{steepness} of a line segment. It measures how much a line rises vertically between two points in a horizontal run. We learn to use a \textbf{slope triangle} to calculate slope values.

There are four types of slope: Positive, Negative, Zero, and Undefined Slope.

In this lesson, we look at \textbf{Positive Slopes}. Line segments that rise from left to right are also called \textit{increasing line segments}. 

To express slope mathematically, we use the formula:
\[ \text{slope} (m) = \frac{\text{rise}}{\text{run}} \]
Here, the slope is denoted by the variable letter \(m\).
\end{mdframed}

\begin{questions}

\question Find the slope using the rise and run?
\begin{multicols}{3}
\begin{parts}
\part \scalebox{0.8}{\linesegment{5}{5}{0}{1}{5}{5}} 
\part \scalebox{0.8}{\linesegment{5}{5}{2}{1}{5}{5}}
\part \scalebox{0.8}{\linesegment{5}{5}{0}{0}{3}{1}} 
% \end{parts}
% \end{multicols}

% \question Find the slope using the rise and run?
% \begin{multicols}{3}
% \begin{parts}
\part \scalebox{0.8}{\linesegment{5}{5}{0}{0}{2}{5}} \smallspace
\part \scalebox{0.8}{\linesegment{5}{5}{0}{3}{5}{5}}\smallspace
\part \scalebox{0.8}{\linesegment{5}{5}{0}{0}{5}{1}} \smallspace
\end{parts}
\end{multicols}

\question 
Construct a line segment to represent a positive slope \(m\).

\begin{multicols}{3}
\begin{parts}
\part \linesegment{5}{5}{1}{1}{4}{4}
\part \linesegment{5}{5}{2}{2}{5}{5}
\part \linesegment{5}{5}{0}{1}{3}{3}
\end{parts}
\end{multicols}

\end{questions}

\end{document}
