



\newcommand{\smallvspace}{\vspace{5mm}}
\newcommand{\normalvspace}{\vspace{10mm}}
\newcommand{\largevspace}{\vspace{15mm}}
\newcommand{\Largevspace}{\vspace{25mm}}
\newcommand{\LARGEvspace}{\vspace{55mm}}
\newcommand{\hugevspace}{\vspace{45mm}}
\newcommand{\Hugevspace}{\vspace{55mm}}
\newcommand{\twoup}{\vspace{75mm}}







\newcommand{\expandbinomialsold}[4]{%
    % Compute intermediate terms (A, B, C)
    \pgfmathparse{int(#1*#3)} \let\A\pgfmathresult  % A = a*c
    \pgfmathparse{int(#1*#4 + #2*#3)} \let\B\pgfmathresult  % B = a*d + b*c
    \pgfmathparse{int(#2*#4)} \let\C\pgfmathresult  % C = b*d

    % Compute absolute values and cast to integer
    \pgfmathparse{int(abs(\A))} \let\absA\pgfmathresult
    \pgfmathparse{int(abs(\B))} \let\absB\pgfmathresult
    \pgfmathparse{int(abs(\C))} \let\absC\pgfmathresult

    % Handle the signs for A, B, C
    \ifnum\A>0 \def\signA{} \else \def\signA{-} \fi
    \ifnum\B>0 \def\signB{+} \else \def\signB{-} \fi
    \ifnum\C>0 \def\signC{+} \else \def\signC{-} \fi

    % Print the quadratic expression in math mode
    \[
    \signA
    \ifnum\A=1
        x^2
    \else
        \pgfmathprintnumber{\A}x^2
    \fi
    \ifnum\B=0
    \else
        \signB
        \ifnum\absB=1
            x
        \else
            \pgfmathprintnumber{\absB}x
        \fi
    \fi
    \ifnum\C=0
    \else
        \signC\pgfmathprintnumber{\absC}
    \fi
    \]  

}




\newcommand{\expandbinomials}[4]{%
    % Compute intermediate terms (A, B, C)
    \pgfmathparse{int(#1*#3)} \let\A\pgfmathresult  % A = a*c
    \pgfmathparse{int(#1*#4 + #2*#3)} \let\B\pgfmathresult  % B = a*d + b*c
    \pgfmathparse{int(#2*#4)} \let\C\pgfmathresult  % C = b*d

    % Compute absolute values and cast to integer
    \pgfmathparse{int(abs(\A))} \let\absA\pgfmathresult
    \pgfmathparse{int(abs(\B))} \let\absB\pgfmathresult
    \pgfmathparse{int(abs(\C))} \let\absC\pgfmathresult

    % Handle the signs for A, B, C
    \ifnum\A>0 \def\signA{} \else \def\signA{-} \fi
    \ifnum\B>0 \def\signB{+} \else \def\signB{-} \fi
    \ifnum\C>0 \def\signC{+} \else \def\signC{-} \fi

    % Print the quadratic expression in inline math mode
    $ \signA
    \ifnum\A=1
        x^2
    \else
        \pgfmathprintnumber{\A}x^2
    \fi
    \ifnum\B=0
    \else
        \signB
        \ifnum\absB=1
            x
        \else
            \pgfmathprintnumber{\absB}x
        \fi
    \fi
    \ifnum\C=0
    \else
        \signC\pgfmathprintnumber{\absC}
    \fi
    $  
}









\newcommand{\productbinomials}[4]{%
    \[
    \left(
    \ifnum#1=1
        x
    \else
        \pgfmathprintnumber{#1}x
    \fi
    \ifnum#2>0
        +\pgfmathprintnumber{#2}
    \else\ifnum#2<0
        -\pgfmathprintnumber{-#2}
    \fi\fi
    \right)
    \left(
    \ifnum#3=1
        x
    \else
        \pgfmathprintnumber{#3}x
    \fi
    \ifnum#4>0
        +\pgfmathprintnumber{#4}
    \else\ifnum#4<0
        -\pgfmathprintnumber{-#4}
    \fi\fi
    \right)
    = 
    \pgfmathparse{#1*#3} \pgfmathprintnumber{\pgfmathresult}x^2
    \pgfmathparse{#1*#4 + #2*#3}
    \ifnum\pgfmathresult>0
        +\pgfmathprintnumber{\pgfmathresult}x
    \else
        \pgfmathprintnumber{\pgfmathresult}x
    \fi
    \pgfmathparse{#2*#4}
    \ifnum\pgfmathresult>0
        +\pgfmathprintnumber{\pgfmathresult}
    \else
        \pgfmathprintnumber{\pgfmathresult}
    \fi
    \]
}



\newcommand{\expandbinomialsequation}[4]{%
    % Compute intermediate terms (A, B, C)
    \pgfmathparse{int(#1*#3)} \let\A\pgfmathresult  % A = a*c
    \pgfmathparse{int(#1*#4 + #2*#3)} \let\B\pgfmathresult  % B = a*d + b*c
    \pgfmathparse{int(#2*#4)} \let\C\pgfmathresult  % C = b*d

    % Compute absolute values and cast to integer
    \pgfmathparse{int(abs(\A))} \let\absA\pgfmathresult
    \pgfmathparse{int(abs(\B))} \let\absB\pgfmathresult
    \pgfmathparse{int(abs(\C))} \let\absC\pgfmathresult

    % Handle the signs for A, B, C
    \ifnum\A>0 \def\signA{} \else \def\signA{-} \fi
    \ifnum\B>0 \def\signB{+} \else \def\signB{-} \fi
    \ifnum\C>0 \def\signC{+} \else \def\signC{-} \fi

    % Print the quadratic expression in math mode with "= 0"
    \[
    \signA
    \ifnum\A=1
        x^2
    \else
        \pgfmathprintnumber{\A}x^2
    \fi
    \ifnum\B=0
    \else
        \signB
        \ifnum\absB=1
            x
        \else
            \pgfmathprintnumber{\absB}x
        \fi
    \fi
    \ifnum\C=0
    \else
        \signC\pgfmathprintnumber{\absC}
    \fi
    = 0
    \]   
}














\newcommand{\slopepoint}[4]{%
    % Convert inputs to integers
    \pgfmathparse{int(#1)}\let\num\pgfmathresult     % Slope numerator as integer
    \pgfmathparse{int(#2)}\let\denom\pgfmathresult   % Slope denominator as integer
    \pgfmathparse{int(#3)}\let\px\pgfmathresult      % x-coordinate of point
    \pgfmathparse{int(#4)}\let\py\pgfmathresult      % y-coordinate of point

    % Determine the slope format (integer or fraction)
    \ifnum\denom=1
        % Display slope as an integer if denominator is 1
        \pgfmathparse{\num}\let\slope\pgfmathresult
    \else
        % Otherwise, display slope as a simplified fraction with proper sign handling
        \ifnum\num<0
            \pgfmathparse{abs(\num)}\let\absnum\pgfmathresult
            \def\slope{-\frac{\absnum}{\denom}}
        \else
            \def\slope{\frac{\num}{\denom}}
        \fi
    \fi

    % Format the (x - h) part
    \def\formatx{%
        \ifnum\px=0 x%
        \else\ifnum\px>0 x - \px%
        \else x + \pgfmathparse{abs(\px)}\pgfmathresult%
        \fi\fi}

    % Format the (y - k) part
    \def\formaty{%
        \ifnum\py=0 y%
        \else\ifnum\py>0 y - \py%
        \else y + \pgfmathparse{abs(\py)}\pgfmathresult%
        \fi\fi}

    % Output the final equation with correct formatting
    \( \formaty = \slope(\formatx) \)
}















\newcommand{\fractosimple}[5]{%
    \ensuremath{%
    % Numerator: Handle #1x and #2 for the left side, ensuring grouping
    \dfrac{%
        \if\relax\detokenize{#1}\relax
        \else
            \ifnum#1=1 x%
            \else\ifnum#1=-1 -x%
            \else \pgfmathprintnumber{#1}x%
            \fi\fi%
        \fi
        \if\relax\detokenize{#2}\relax
        \else
            \if\relax\detokenize{#1}\relax \pgfmathprintnumber{#2}%
            \else \ifnum#2<0 \pgfmathprintnumber{#2}%
            \else +\pgfmathprintnumber{#2}%
            \fi\fi%
        \fi%
    }{#3} = %
    % Right side: Handle the terms #4x + #5
    \if\relax\detokenize{#4}\relax
    \else
        \ifnum#4=1 x%
        \else\ifnum#4=-1 -x%
        \else \pgfmathprintnumber{#4}x%
        \fi\fi%
    \fi
    \if\relax\detokenize{#5}\relax
    \else
        \if\relax\detokenize{#4}\relax \pgfmathprintnumber{#5}%
        \else \ifnum#5<0 \pgfmathprintnumber{#5}%
        \else +\pgfmathprintnumber{#5}%
        \fi\fi%
    \fi%
    }
}










% Define the custom command for the fractional equation
\newcommand{\fractosimple}[5]{%
    \ensuremath{%
    \dfrac{%
        \ifnum#1=1 x%
        \else\ifnum#1=-1 -x%
        \else \pgfmathprintnumber{#1}x%
        \fi\fi%
        \ifnum#2<0 \pgfmathprintnumber{#2}%
        \else +\pgfmathprintnumber{#2}%
        \fi%
    }{#3} = %
    \ifnum#4=1 x%
    \else\ifnum#4=-1 -x%
    \else \pgfmathprintnumber{#4}x%
    \fi\fi%
    \ifnum#5<0 \pgfmathprintnumber{#5}%
    \else +\pgfmathprintnumber{#5}%
    \fi%
    }
}










\newcommand{\linearsimplified}[4]{%
    \ensuremath{%
    % First term on the left side (handles empty case)
    \if\relax\detokenize{#1}\relax
    \else
        \ifnum#1=1 x
        \else\ifnum#1=-1 -x
        \else \pgfmathprintnumber{#1}x
        \fi\fi
    \fi
    % Second term on the left side (constant) - Print without '+' if no x-term
    \if\relax\detokenize{#2}\relax
    \else
        \if\relax\detokenize{#1}\relax
            \pgfmathprintnumber{#2}
        \else
            \ifnum#2<0 \pgfmathprintnumber{#2}
            \else +\pgfmathprintnumber{#2}
            \fi
        \fi
    \fi
    = % Equal sign between left and right sides
    % First term on the right side (handles empty case)
    \if\relax\detokenize{#3}\relax
    \else
        \ifnum#3=1 x
        \else\ifnum#3=-1 -x
        \else \pgfmathprintnumber{#3}x
        \fi\fi
    \fi
    % Second term on the right side (constant) - Print without '+' if no x-term
    \if\relax\detokenize{#4}\relax
    \else
        \if\relax\detokenize{#3}\relax
            \pgfmathprintnumber{#4}
        \else
            \ifnum#4<0 \pgfmathprintnumber{#4}
            \else +\pgfmathprintnumber{#4}
            \fi
        \fi
    \fi
    }
}


\newcommand{\linearsimplifiedolder}[4]{%
    \ensuremath{%
    % First term on the left side (handles empty case)
    \if\relax\detokenize{#1}\relax
    \else
        \ifnum#1=1 x
        \else\ifnum#1=-1 -x
        \else \pgfmathprintnumber{#1}x
        \fi\fi
    \fi
    % Second term on the left side (constant)
    \if\relax\detokenize{#2}\relax
    \else
        \ifnum#2<0 \pgfmathprintnumber{#2}
        \else +\pgfmathprintnumber{#2}
        \fi
    \fi
    = % Equal sign between left and right sides
    % First term on the right side (handles empty case)
    \if\relax\detokenize{#3}\relax
    \else
        \ifnum#3=1 x
        \else\ifnum#3=-1 -x
        \else \pgfmathprintnumber{#3}x
        \fi\fi
    \fi
    % Second term on the right side (constant)
    \if\relax\detokenize{#4}\relax
    \else
        \ifnum#4<0 \pgfmathprintnumber{#4}
        \else +\pgfmathprintnumber{#4}
        \fi
    \fi
    }
}


% Define the linear equation command
\newcommand{\linearsimplifiedold}[4]{%
    \ensuremath{%
    \ifnum#1=1
        x
    \else\ifnum#1=-1
        -x
    \else
        \pgfmathprintnumber{#1}x
    \fi\fi
    \ifnum#2<0
        \pgfmathprintnumber{#2}
    \else
        +\pgfmathprintnumber{#2}
    \fi
    = 
    \ifnum#3=1
        x
    \else\ifnum#3=-1
        -x
    \else
        \pgfmathprintnumber{#3}x
    \fi\fi
    \ifnum#4<0
        \pgfmathprintnumber{#4}
    \else
        +\pgfmathprintnumber{#4}
    \fi
    }
}








\newcommand{\circleEquation}[3]{%
    \ensuremath{%
    \left( x%
    \ifnum#1>0 - #1%
    \else\ifnum#1<0 + \pgfmathtruncatemacro{\absX}{abs(#1)}\absX%
    \fi\fi \right)^2 + \left( y%
    \ifnum#2>0 - #2%
    \else\ifnum#2<0 + \pgfmathtruncatemacro{\absY}{abs(#2)}\absY%
    \fi\fi \right)^2 = #3^2%
    }%
}





\newcommand{\circleConst}[3]{%
    \ensuremath{%
    \ifnum#1=0 x^2 \else \left( x%
    \ifnum#1>0 - #1%
    \else + \pgfmathtruncatemacro{\absX}{abs(#1)}\absX%
    \fi \right)^2 \fi%
    + \ifnum#2=0 y^2 \else \left( y%
    \ifnum#2>0 - #2%
    \else + \pgfmathtruncatemacro{\absY}{abs(#2)}\absY%
    \fi \right)^2 \fi%
    = #3%
    }%
}







\newcommand{\circleRsquared}[3]{%
    \ensuremath{%
    \ifnum#1=0 x^2 \else \left( x%
    \ifnum#1>0 - #1%
    \else + \pgfmathtruncatemacro{\absX}{abs(#1)}\absX%
    \fi \right)^2 \fi%
    + \ifnum#2=0 y^2 \else \left( y%
    \ifnum#2>0 - #2%
    \else + \pgfmathtruncatemacro{\absY}{abs(#2)}\absY%
    \fi \right)^2 \fi%
    = #3^2%
    }%
}



\newcommand{\circleRsquaredold}[3]{%
    \ensuremath{%
    \ifnum#1=0 x^2 \else \left( x%
    \ifnum#1>0 - #1%
    \else + \pgfmathprintnumber{abs(#1)}%
    \fi \right)^2 \fi%
    + \ifnum#2=0 y^2 \else \left( y%
    \ifnum#2>0 - #2%
    \else + \pgfmathprintnumber{abs(#2)}%
    \fi \right)^2 \fi%
    = #3^2%
    }%
}


% Define a custom command for circle equation without squared constant
\newcommand{\circleAtOrigin}[1]{%
    \( x^2 + y^2 = #1 \)%
}









\newcommand{\circleAtOriginRsquared}[1]{%
    \( x^2 + y^2 = #1^2 \)%
}



\newcommand{\elimination}[6]{%
    % Compute constants c and p based on given coefficients and x, y values
    \pgfmathsetmacro{\cval}{#3*#1 + #4*#2} % c = a*x_value + b*y_value
    \pgfmathsetmacro{\pval}{#5*#1 + #6*#2} % p = d*x_value + e*y_value

    % Format coefficients (removing 1 or -1)
    \def\formata{%
        \ifnum#3=1
            x
        \else\ifnum#3=-1
            -x
        \else
            \pgfmathprintnumber{#3}x
        \fi\fi}

    \def\formatb{%
        \ifnum#4=1
            +y
        \else\ifnum#4=-1
            -y
        \else
            \ifnum#4<0 \pgfmathprintnumber{#4}y \else +\pgfmathprintnumber{#4}y \fi
        \fi\fi}

    \def\formatd{%
        \ifnum#5=1
            x
        \else\ifnum#5=-1
            -x
        \else
            \pgfmathprintnumber{#5}x
        \fi\fi}

    \def\formate{%
        \ifnum#6=1
            +y
        \else\ifnum#6=-1
            -y
        \else
            \ifnum#6<0 \pgfmathprintnumber{#6}y \else +\pgfmathprintnumber{#6}y \fi
        \fi\fi}

    % Display equations in aligned format to ensure = signs are aligned
    $\left\{ \begin{aligned}
    \formata \formatb &= \pgfmathprintnumber{\cval} \\[-1pt]
    \formatd \formate &= \pgfmathprintnumber{\pval}
    \end{aligned} \right.$
}


\newcommand{\eliminationold}[6]{%
    % Compute constants c and p based on given coefficients and x, y values
    \pgfmathsetmacro{\cval}{#3*#1 + #4*#2} % c = a*x_value + b*y_value
    \pgfmathsetmacro{\pval}{#5*#1 + #6*#2} % p = d*x_value + e*y_value

    % Format coefficients (removing 1 or -1)
    \def\formata{%
        \ifnum#3=1
            x
        \else\ifnum#3=-1
            -x
        \else
            \pgfmathprintnumber{#3}x
        \fi\fi}

    \def\formatb{%
        \ifnum#4=1
            +y
        \else\ifnum#4=-1
            -y
        \else
            \ifnum#4<0 \pgfmathprintnumber{#4}y \else +\pgfmathprintnumber{#4}y \fi
        \fi\fi}

    \def\formatd{%
        \ifnum#5=1
            x
        \else\ifnum#5=-1
            -x
        \else
            \pgfmathprintnumber{#5}x
        \fi\fi}

    \def\formate{%
        \ifnum#6=1
            +y
        \else\ifnum#6=-1
            -y
        \else
            \ifnum#6<0 \pgfmathprintnumber{#6}y \else +\pgfmathprintnumber{#6}y \fi
        \fi\fi}

    % Display equations in a vertically stacked array format
    \[
    \begin{array}{c}
    \formata \formatb = \pgfmathprintnumber{\cval} \\[6pt]
    \formatd \formate = \pgfmathprintnumber{\pval}
    \end{array}
    \]
}













\newcommand{\constructline}[1]{% #1=grid size
\begin{tikzpicture}
    % Draw a grid with size specified by #1 x #1
    \draw[step=1cm, gray, very thin] (0,0) grid (#1,#1);
\end{tikzpicture}

% Generate random numerator and denominator for the slope
\pgfmathsetmacro{\numerator}{int(random(-#1,#1))}
\pgfmathsetmacro{\denominator}{int(random(-#1,#1))}
% Ensure the denominator is not zero
\whileif{\denominator == 0}{
    \pgfmathsetmacro{\denominator}{int(random(-#1,#1))}
}

% Print the slope statement below the grid
\vspace{10pt} % Adjust space as needed
\noindent Slope: $slope=\dfrac{\numerator}{\denominator}$
}