\documentclass{beamer}
\usepackage{amsmath}
\usepackage{array}
\usepackage{multicol}
\usepackage{booktabs}

\usetheme{Singapore}

% Change alert color to red and apply huge bold font
\setbeamercolor{alerted text}{fg=red}
\setbeamerfont{alerted text}{series=\bfseries,size=\Huge}

% Font Size for frametitle and items
\setbeamerfont{frametitle}{size=\huge}
\setbeamerfont{itemize/enumerate body}{size=\large}
\setbeamerfont{itemize/enumerate subbody}{size=\large}

% Title Page
\title{\huge Square Roots, General Roots, and Fractional Powers}
\author{Mr. Downes}
\date{}

\begin{document}

% Title Slide
\begin{frame}
  \titlepage
\end{frame}

% Slide 1: Learning Intentions
\begin{frame}{\huge Learning Intentions}
  \begin{itemize}[<+-| alert@+>]
    \item Look up, recall, understand, and apply the identities for roots and powers.
    \item Work with square roots, cube roots, and fractional exponents.
  \end{itemize}
\end{frame}

% Slide 2: Roots - Notation and Fractional Powers
\begin{frame}{\huge Roots and Fractional Powers}
\begin{itemize}
    \item \textbf{Roots - Notation, Meaning, and Unit Fraction Notation:} 
    \[
    a^{\frac{1}{q}} = \sqrt[q]{a}
    \]
    \item \textbf{General Fractional Exponent as a Root Taken to a Power:}
    \[
    a^{\frac{p}{q}} = \sqrt[q]{a^p} = \left( \sqrt[q]{a} \right)^p
    \]
    \item \textbf{Finding Unknown Exponents Using Logarithms:} 
    \[
    a^x = y \iff \log_a y = x
    \]
\end{itemize}
\end{frame}

% Slide 3: Powers Table
\begin{frame}{\huge Powers Table (Squares and Cubes)}
\centering
\begin{tabular}{|c|c|c|c|c|c|c|c|c|c|}
\hline
$2^2 = 4$   & $3^2 = 9$   & $4^2 = 16$  & $5^2 = 25$  & $6^2 = 36$  & $7^2 = 49$  & $8^2 = 64$  & $9^2 = 81$  & $10^2 = 100$ & $11^2 = 121$ \\ \hline
$2^3 = 8$   & $3^3 = 27$  & $4^3 = 64$  & $5^3 = 125$ & $6^3 = 216$ & $7^3 = 343$ & $8^3 = 512$ & $9^3 = 729$ & $10^3 = 1000$ & $11^3 = 1331$ \\ \hline
\end{tabular}
\end{frame}

% Slide 4: Roots Table
\begin{frame}{\huge Roots Table}
\centering
\begin{tabular}{|c|c|c|c|c|c|c|c|c|}
\hline
$\sqrt{4} = 2$   & $\sqrt{9} = 3$   & $\sqrt{16} = 4$  & $\sqrt{25} = 5$  & $\sqrt{36} = 6$  & $\sqrt{49} = 7$  & $\sqrt{64} = 8$  & $\sqrt{81} = 9$  & $\sqrt{100} = 10$ \\ \hline
$\sqrt[3]{8} = 2$   & $\sqrt[3]{27} = 3$  & $\sqrt[3]{64} = 4$  & $\sqrt[3]{125} = 5$ & $\sqrt[3]{216} = 6$ & $\sqrt[3]{343} = 7$ & $\sqrt[3]{512} = 8$ & $\sqrt[3]{729} = 9$ & $\sqrt[3]{1000} = 10$  \\ \hline
\end{tabular}
\end{frame}

% Slide 5: Example Problem (Square Root)
\begin{frame}{\huge Example: Square Roots}
  \textbf{The square root of a number is the number that, when multiplied by itself, gives the original number.}
  \[
  a^{\frac{1}{2}} = \sqrt{a}
  \]
  \textbf{Example:} The square root of 16 is 4, because \(4 \times 4 = 16\). So, \(\sqrt{16} = 4\).
\end{frame}

% Slide 6: Square Roots Task
\begin{frame}{\huge Square Roots Practice}
Use the $\sqrt{\blacksquare}$ function to find the following square roots:
\begin{multicols}{3}
\begin{itemize}
    \item $\sqrt{4}$
    \item $\sqrt{25}$
    \item $\sqrt{81}$
    \item $\sqrt{361}$
    \item $\sqrt{484}$
    \item $\sqrt{9801}$
\end{itemize}
\end{multicols}
\end{frame}

% Slide 7: Cube Roots Task
\begin{frame}{\huge Cube Roots Practice}
Write down the value of the following cube roots:
\begin{multicols}{3}
\begin{itemize}
    \item $\sqrt[3]{1331}$
    \item $\sqrt[3]{343}$
    \item $\sqrt[3]{64}$
    \item $\sqrt[3]{216}$
    \item $\sqrt[3]{1000}$
    \item $\sqrt[3]{512}$
\end{itemize}
\end{multicols}
\end{frame}

% Slide 8: Homework
\begin{frame}{\huge Homework}
Complete the remaining tasks:
\begin{itemize}
    \item Memorize the square and cube roots.
    \item Practice finding the value of powers and roots from the tables provided.
    \item Finish exercises on page 330.
\end{itemize}
\end{frame}

\end{document}
