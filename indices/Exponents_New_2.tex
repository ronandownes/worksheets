
\documentclass[12pt, a4paper, addpoints]{exam}

\title{Zero and negative Index Rules worksheet}


\pagestyle{empty} % Suppress page numbers
\usepackage[top=5mm, bottom=20mm, left=11mm, right=11mm]{geometry}
\usepackage{amsmath} % For mathematical symbols
\usepackage{multicol} % For multi-column layout
\usepackage{tabularx}
\usepackage{pgfmath} % For math parsing and calculations
\usepackage{xcolor} % For color customization
\pagestyle{empty}
% \author{Mr Downes}
\newcommand{\ts}{\vspace{16mm}}
\newcommand{\ms}{\vspace{20mm}}
\newcommand{\bs}{\vspace{30mm}}
\newcommand{\ls}{\vspace{30mm}}
\newcommand{\hs}{\vspace{30mm}}
\usepackage{array}     % For centering text vertically
\usepackage{graphicx}  % For rotating text
\title{Zero and negative Index Rules worksheet}
\date{}
\begin{document}
% Adjust vertical spacing before and after the title

\maketitle
\vspace{-28mm}
% \section{Calculations using powers  $a^p$  and $a^q$ }
 % \huge


\begin{questions}



\question Recall and  understanding these indices and logarithms identities from page 21.

% \begin{multicols}{2}
\begin{parts}
\part \text{Product of Powers — Add the exponents} \hfill $a^p a^q = a^{p+q}$

\part \text{Quotient of Powers — Subtract  the exponents} \hfill $\dfrac{a^p}{a^q} = a^{p-q}$
\part \text{Power of a Power — Multiply the exponents} \hfill $(a^p)^q = a^{pq}$

\part \text{Zero Exponent — Any non-zero base raised to zero equals 1:} \hfill $a^0 = 1$

\part \text{Negative Exponent — Recipricate  and change the sign} \hfill $a^{-p} = \dfrac{1}{a^p}$


% \part \text{Power of a Quotient} \hfill $\left( \dfrac{a}{b} \right)^p = \frac{a^p}{b^p}$

\part \text{Power of a Product — Distribute the exponent} \hfill $(ab)^p = a^p b^p$



% \part \text{Unit Fraction to root  Form:} \hfill $a^{\frac{1}{q}} = \sqrt[q]{a}$

% \part \text{ General Exponent Form} \hfill $a^{\frac{p}{q}} = \sqrt[q]{a^p} = \left( \sqrt[q]{a} \right)^p$
\(\(\)\)
% \part \text{Exponent to  Logarithmic Form} \\ \hfill $a^x = y \iff \log_a y = x$
\end{parts}
% \end{multicols}

\large

\question Apply the Product of Powers rule — Add the exponents to simplify.

\setlength{\columnsep}{20pt}
\begin{multicols}{6}
\begin{parts}
\part $x^3 x^5$
\part $y^4 y^6$
\part $a^2 a^7$
\part $b^3 b^9$
\part $m^5 m^2$
\part $z^4 z^8$
\end{parts}
\end{multicols}
\ts
\question Subtract the exponents to apply the Quotient of Powers rule.

\setlength{\columnsep}{20pt}
\begin{multicols}{6}
\begin{parts}
\part $\dfrac{3^6}{3^2}$
\part $\dfrac{2^6}{y^6}$
\part $\dfrac{a^7}{a^3}$
\part $\dfrac{b^8}{b^4}$
\part $\dfrac{m^5}{m^2}$
\part $\dfrac{z^9}{z^6}$
\end{parts}
\end{multicols}


\ts







\question  Recipricate  and change the sign to give positive indices in $\dfrac{1}{a^n}$  form.
\setlength{\columnsep}{20pt}
\begin{multicols}{6}
\begin{parts}
\part \(7^{-6}\)

\part \(4^{-1} \)

\part \(5^{-8}\) 

\part \(x^{-p}\) 

\part \(y^{-q} \)

\part \(z^{-0} \)

\end{parts}
\end{multicols}
\ts



\question Recipricate  and change the sign to give positive indices in $a^n$ form

\setlength{\columnsep}{20pt}
\begin{multicols}{6}
\begin{parts}
\part $\dfrac{1}{7^{-3}}$
\part $\dfrac{1}{3^{-4}}$
\part $\dfrac{1}{4^{-1}}$
\part $\dfrac{1}{5^{-4}}$
\part $\dfrac{1}{x^{-p}}$
\part $\dfrac{1}{0^{-2}}$
\end{parts}
\end{multicols}
\ts


\newpage
\question Remove the brackets by distributing the exponents  \((ab)^p = a^p b^p\).

\setlength{\columnsep}{20pt}
\begin{multicols}{5}
\begin{parts}
\part $(ab)^3 $
\part $(xy)^4 $
\part $(2x)^6$
\part $(2y)^4 $
\part $(3mn)^2 $


\end{parts}
\end{multicols}
\verticalspace

\ts


\question Power of a Power — Multiply the exponents
\setlength{\columnsep}{20pt} 
\begin{multicols}{6}
\begin{parts} 
\part $\left( 3^4 \right)^2$ \ms
\part $\left( x^5 \right)^3$ \ms
\part $\left( a^2 \right)^4$ \ms
\part $\left( 7^3 \right)^2$ \ms
\part $\left( m^6 \right)^2$ \ms
\part $\left( b^3 \right)^4$ 
\end{parts}
\end{multicols}

\ts



% \question Continue and familiarise yourself with the doubling number pattern.

% % Adjust vertical spacing within the table rows
% \renewcommand{\arraystretch}{2.5}

% \begin{center}
% \begin{tabularx}{\linewidth}{|>{\centering\arraybackslash}c|*{12}{>{\centering\arraybackslash}X|}}
% \hline
% $2^n$ & $2^1$ & $2^2$ & $2^3$ & $2^4$ & $2^5$ & $2^6$ & $2^7$ & $2^8$ & $2^9$ & $2^{10}$ & $2^{11}$ & $2^{12}$ \\
% \hline
% $2^n$ & 2 & 4 & 8 &  &  &  &  &  &  &  &  & \makebox[0pt][l]{\raisebox{-1.5ex}{\rotatebox{90}{4096}}}  \\
% \hline
% \end{tabularx}
% \end{center}
% \ts



% \question Continue and familiarise yourself with the doubling number pattern.

% % Adjust vertical spacing within the table rows
% \renewcommand{\arraystretch}{2.5}

% \begin{center}
% \begin{tabularx}{\linewidth}{|>{\centering\arraybackslash}c|*{13}{>{\centering\arraybackslash}X|}}
% \hline
% $n$ & $2^0$& $2^1$ & $2^2$ & $2^3$ & $2^4$ & $2^5$ & $2^6$ & $2^7$ & $2^8$ & $2^9$ & $2^{10}$ & $2^{11}$ & $2^{12}$ \\
% \hline
% $2^n$ &1& 2 & 4 & 8 & 16 & 32 & 64 & 128 & 256 & 512 & 1024 & 2048 & \makebox[0pt][l]{\raisebox{-1.5ex}{\rotatebox{90}{4096}}}  \\
% \hline
% \end{tabularx}
% \end{center}
% \ts
\question  Reproduce the table from mental maths in your notes. Graph the exponential functions  \(2^x\)
% , \(3^x\), \(5^x\),\( x^2 \)and \(x^3\) 
on Desmos and explore how the graph  behaves for positive and negative values of \(x\). Does it have a minimum value like \(x^2\) has?


% Adjust vertical spacing within the table rows
\renewcommand{\arraystretch}{2.5}

\begin{center}
\begin{tabularx}{\linewidth}{|>{\centering\arraybackslash}c|*{17}{>{\centering\arraybackslash}X|}}
\hline
$n$ & $2^{-8}$ & $2^{-7}$ & $2^{-6}$ & $2^{-5}$ & $2^{-4}$ & $2^{-3}$ & $2^{-2}$ & $2^{-1}$ & $2^0$ & $2^1$ & $2^2$ & $2^3$ & $2^4$ & $2^5$ & $2^6$ & $2^7$ & $2^8$ \\
\hline
$2^n$ & $\dfrac{1}{256}$ & $\dfrac{1}{128}$ & $\dfrac{1}{64}$ & $\dfrac{1}{32}$ & $\dfrac{1}{16}$ & $\dfrac{1}{8}$ & $\dfrac{1}{4}$ & $\dfrac{1}{2}$ & 1 & 2 & 4 & 8 & 16 & 32 & 64 & 128 & 256 \\
\hline
\end{tabularx}
\end{center}












% \part 

% \begin{center}
% \begin{tabularx}{\linewidth}{|>{\centering\arraybackslash}c|*{13}{>{\centering\arraybackslash}X|}}
% \hline
% $n$ & $3^{-6}$ & $3^{-5}$ & $3^{-4}$ & $3^{-3}$ & $3^{-2}$ & $3^{-1}$ & $3^0$ & $3^1$ & $3^2$ & $3^3$ & $3^4$ & $3^5$ & $3^6$ \\
% \hline
% $3^n$ & $\dfrac{1}{729}$ & $\dfrac{1}{243}$ & $\dfrac{1}{81}$ & $\dfrac{1}{27}$ & $\dfrac{1}{9}$ & $\dfrac{1}{3}$ & 1 & 3 & 9 & 27 & 81 & 243 & 729 \\
% \hline
% \end{tabularx}
% \end{center}
% \ts

% \part 

% \begin{center}
% \begin{tabularx}{\linewidth}{|>{\centering\arraybackslash}c|*{12}{>{\centering\arraybackslash}X|}}
% \hline
% $n$  & $5^{-5}$ & $5^{-4}$ & $5^{-3}$ & $5^{-2}$ & $5^{-1}$ & $5^0$ & $5^1$ & $5^2$ & $5^3$ & $5^4$ & $5^5$ & $5^6$ \\
% \hline
% $5^n$ & $\dfrac{1}{3125}$ & $\dfrac{1}{625}$ & $\dfrac{1}{125}$ & $\dfrac{1}{25}$ & $\dfrac{1}{5}$ & 1 & 5 & 25 & 125 & 625 & 3125 & 15625 \\
% \hline
% \end{tabularx}
% \end{center}



% \end{parts}




% \question Complete the table of squares up to \(16^2\) using your calculator. Notice the pattern in what is added at each step (i.e., the sequence of odd integers).

% \begin{center}
% \begin{tabularx}{\linewidth}{|>{\centering\arraybackslash}c|*{17}{>{\centering\arraybackslash}X|}}
% \hline
% $n^2$& $0^2$ & $1^2$ & $2^2$ & $3^2$ & $4^2$ & $5^2$ & $6^2$ & $7^2$ & $8^2$ & $9^2$ & $10^2$ & $11^2$ & $12^2$ & $13^2$ & $14^2$ & $15^2$ & $16^2$ \\
% \hline

% Value & 0 & 1 & 4 & 9 & 16 & 25 & 36 & 49 & 64 & 81 & 100 & 121 & 144 & 169 & 196 & 225 & 256 \\\hline
% Add  & 1 & 3 & 5 & 7 & 9 & 11 & 13 & 15 & 17 & 19 & 21 & 23 & 25 & 27 & 29 & 31&33 \\\hline


% \hline
% \end{tabularx}
% \end{center}













%  \question Complete the table of squares up to \(16^2\) using your calculator.

% \begin{center}
% \begin{tabularx}{\linewidth}{|>{\centering\arraybackslash}c|*{16}{>{\centering\arraybackslash}X|}}
% \hline
% $n$ & $1^2$ & $2^2$ & $3^2$ & $4^2$ & $5^2$ & $6^2$ & $7^2$ & $8^2$ & $9^2$ & $10^2$ & $11^2$ & $12^2$ & $13^2$ & $14^2$ & $15^2$ & $16^2$ \\
% \hline
% $n^2$ & 1 & 4 & 9 & 16 & 25 & 36 & 49 & 64 & 81 & 100 & 121 & 144 & 169 & 196 & 225 & 256 \\
% \hline
% \end{tabularx}
% \end{center}


% \question Complete the table of cubes up to \(11^3\) using your calculator.

% \begin{center}
% \begin{tabularx}{\linewidth}{|>{\centering\arraybackslash}c|*{11}{>{\centering\arraybackslash}X|}}
% \hline
% $n$ & $1^3$ & $2^3$ & $3^3$ & $4^3$ & $5^3$ & $6^3$ & $7^3$ & $8^3$ & $9^3$ & $10^3$ & $11^3$ \\
% \hline
% $n^3$ & 1 & 8 & 27 & 64 & 125 & 216 & 343 & 512 & 729 & 1000 & 1331 \\
% \hline
% \end{tabularx}
% \end{center}
























% \ts
% \question Continue the tripling number pattern.

% \begin{center}
% \begin{tabularx}{\linewidth}{|>{\centering\arraybackslash}c|*{12}{>{\centering\arraybackslash}X|}}
% \hline
% $n$ & $3^1$ & $3^2$ & $3^3$ & $3^4$ & $3^5$ & $3^6$ & $3^7$ & $3^8$ & $3^9$ & $3^{10}$  \\
% \hline
% $3^n$ & 3 & 9 & 27 & 81 & 243 & 729 & 2187 & 6561 & 19683 &\rotatebox{90}{ 59049} \\
% \hline
% \end{tabularx}
% \end{center}
% \ts

% \question Continue the powers of $5$ number pattern.

% \begin{center}
% \begin{tabularx}{\linewidth}{|>{\centering\arraybackslash}c|*{12}{>{\centering\arraybackslash}X|}}
% \hline
% $n$  & $5^{-5}$ & $5^{-4}$ & $5^{-3}$ & $5^{-2}$ & $5^{-1}$ & $5^0$ & $5^1$ & $5^2$ & $5^3$ & $5^4$ & $5^5$ & $5^6$ \\
% \hline
% $5^n$ & $\dfrac{1}{3125}$ & $\dfrac{1}{625}$ & $\dfrac{1}{125}$ & $\dfrac{1}{25}$ & $\dfrac{1}{5}$ & 1 & 5 & 25 & 125 & 625 & 3125 & 15625 \\
% \hline
% \end{tabularx}
% \end{center}
% \ts

% \question Complete the table of squares up to \(16^2\).

% \begin{center}
% \begin{tabularx}{\linewidth}{|>{\centering\arraybackslash}c|*{16}{>{\centering\arraybackslash}X|}}
% \hline
% $n$ & $1^2$ & $2^2$ & $3^2$ & $4^2$ & $5^2$ & $6^2$ & $7^2$ & $8^2$ & $9^2$ & $10^2$ & $11^2$ & $12^2$ & $13^2$ & $14^2$ & $15^2$ & $16^2$ \\
% \hline
% $n^2$ & 1 & 4 & 9 & 16 & 25 & 36 & 49 & 64 & 81 & 100 & 121 & 144 & 169 & 196 & 225 & 256 \\
% \hline
% \end{tabularx}
% \end{center}
% \ts

% \question Complete the table of cubes up to \(11^3\).

% \begin{center}
% \begin{tabularx}{\linewidth}{|>{\centering\arraybackslash}c|*{11}{>{\centering\arraybackslash}X|}}
% \hline
% $n$ & $1^3$ & $2^3$ & $3^3$ & $4^3$ & $5^3$ & $6^3$ & $7^3$ & $8^3$ & $9^3$ & $10^3$ & $11^3$ \\
% \hline
% $n^3$ & 1 & 8 & 27 & 64 & 125 & 216 & 343 & 512 & 729 & 1000 & 1331 \\
% \hline
% \end{tabularx}
% \end{center}
% \ts





% Question 1: Simple products of powers
\question  Homework: Simplify the following expressions avoiding negative exponents.

\begin{multicols}{5}
\begin{parts}
\part $x^2 \cdot x^5$
\part $y^3 \cdot y^4$
\part $b^5 \cdot b^7$
\part $c \cdot c^3$
\part $m^4 \cdot m^6$
\part $x^9 \cdot x$
\part $\dfrac{x^8}{x^3}$
\part $\dfrac{a^7}{a^{11}}$
\part $\dfrac{ab}{b^6 \cdot a^2}$
\part $\dfrac{m^5}{m^2}$
\part $\dfrac{x^{10}}{x^6}$
\part $\dfrac{z^7}{z^5}$
\part $\dfrac{c^8}{c^3}$
\part $(y^4)^2$
\part $(a^3)^4$
\part $(b^5)^2$
\part $(m^2)^5$

\part $(z^6)^2$
\part $(c^7)^2$
\part $x^{-3}$

\part $\dfrac{1}{a^{-2}}$
\part $\dfrac{1}{b^{-4}}$
\part $m^{-6}$
\part $\dfrac{1}{x^{-1}}$
\part $\dfrac{1}{z^{-2}}$
\part $c^{-7}$
\end{parts}
\end{multicols}


% Final question: Mixed and challenging expressions
\question Homework Mixed and challenging expressions:
Simplify the following expressions. Use all exponent rules where necessary.

\begin{multicols}{4}
\begin{parts}
\part $\dfrac{7x^2 \cdot y^{-5}}{2x^{-2} \cdot y^3}$
\part $\dfrac{4x \cdot y^5}{8x^5 \cdot y^5}$
\part $\left( \dfrac{6x^2 \cdot y^{-2}}{2x^{-5} \cdot y^{-2}} \right)^5$
\part $\dfrac{4x^{-2} \cdot y^2}{4x^2 \cdot y}$
\part $\dfrac{10x^3 \cdot y^{-2}}{7x^2 \cdot y^2}$
\part $\left( \dfrac{3x^0 \cdot y^{-5}}{6x^2 \cdot y^0} \right)^3$
\part $\dfrac{5x^2 \cdot y^{-4}}{10x^{-2} \cdot y^{-1}}$
\part $\dfrac{8x^{-2} \cdot y^2}{10x^{-1} \cdot y^{-2}}$
\part $\left( \dfrac{5x^4 \cdot y^4}{2x^2 \cdot y^3} \right)^2$
\part $\dfrac{8x^5 \cdot y^{-3}}{3x^{-5} \cdot y^{-2}}$
\part $\dfrac{2x^5 \cdot y^{-3} \cdot x^2}{x^2 \cdot 3y^4}$
\part $\dfrac{a^7 \cdot b^3}{a^{-2} \cdot 5b^5}$


\end{parts}
\end{multicols}


\end{questions}


\end{document}
