\documentclass[12pt, a4paper, addpoints]{exam}
\pagestyle{empty} % Suppress page numbers

\usepackage[margin=15mm]{geometry} % Set all margins to 15mm
\usepackage{amsmath} % For mathematical symbols
\usepackage{multicol} % For multi-column layout
\usepackage{pgfmath} % For math parsing and calculations
\usepackage{xcolor} % For color customization
\usepackage[margin=15mm]{geometry}
% Define the \expandbinomials command



\newcommand{\smallvspace}{\vspace{5mm}}
\newcommand{\normalvspace}{\vspace{10mm}}
\newcommand{\largevspace}{\vspace{15mm}}
\newcommand{\Largevspace}{\vspace{25mm}}
\newcommand{\LARGEvspace}{\vspace{55mm}}
\newcommand{\hugevspace}{\vspace{45mm}}
\newcommand{\Hugevspace}{\vspace{55mm}}
\newcommand{\twoup}{\vspace{75mm}}

\newcommand{\expandbinomialsold}[4]{%
    % Compute intermediate terms (A, B, C)
    \pgfmathparse{int(#1*#3)} \let\A\pgfmathresult  % A = a*c
    \pgfmathparse{int(#1*#4 + #2*#3)} \let\B\pgfmathresult  % B = a*d + b*c
    \pgfmathparse{int(#2*#4)} \let\C\pgfmathresult  % C = b*d

    % Compute absolute values and cast to integer
    \pgfmathparse{int(abs(\A))} \let\absA\pgfmathresult
    \pgfmathparse{int(abs(\B))} \let\absB\pgfmathresult
    \pgfmathparse{int(abs(\C))} \let\absC\pgfmathresult

    % Handle the signs for A, B, C
    \ifnum\A>0 \def\signA{} \else \def\signA{-} \fi
    \ifnum\B>0 \def\signB{+} \else \def\signB{-} \fi
    \ifnum\C>0 \def\signC{+} \else \def\signC{-} \fi

    % Print the quadratic expression in math mode
    \[
    \signA
    \ifnum\A=1
        x^2
    \else
        \pgfmathprintnumber{\A}x^2
    \fi
    \ifnum\B=0
    \else
        \signB
        \ifnum\absB=1
            x
        \else
            \pgfmathprintnumber{\absB}x
        \fi
    \fi
    \ifnum\C=0
    \else
        \signC\pgfmathprintnumber{\absC}
    \fi
    \]  

}




\newcommand{\expandbinomials}[4]{%
    % Compute intermediate terms (A, B, C)
    \pgfmathparse{int(#1*#3)} \let\A\pgfmathresult  % A = a*c
    \pgfmathparse{int(#1*#4 + #2*#3)} \let\B\pgfmathresult  % B = a*d + b*c
    \pgfmathparse{int(#2*#4)} \let\C\pgfmathresult  % C = b*d

    % Compute absolute values and cast to integer
    \pgfmathparse{int(abs(\A))} \let\absA\pgfmathresult
    \pgfmathparse{int(abs(\B))} \let\absB\pgfmathresult
    \pgfmathparse{int(abs(\C))} \let\absC\pgfmathresult

    % Handle the signs for A, B, C
    \ifnum\A>0 \def\signA{} \else \def\signA{-} \fi
    \ifnum\B>0 \def\signB{+} \else \def\signB{-} \fi
    \ifnum\C>0 \def\signC{+} \else \def\signC{-} \fi

    % Print the quadratic expression in inline math mode
    $ \signA
    \ifnum\A=1
        x^2
    \else
        \pgfmathprintnumber{\A}x^2
    \fi
    \ifnum\B=0
    \else
        \signB
        \ifnum\absB=1
            x
        \else
            \pgfmathprintnumber{\absB}x
        \fi
    \fi
    \ifnum\C=0
    \else
        \signC\pgfmathprintnumber{\absC}
    \fi
    $  
}









\newcommand{\productbinomials}[4]{%
    \[
    \left(
    \ifnum#1=1
        x
    \else
        \pgfmathprintnumber{#1}x
    \fi
    \ifnum#2>0
        +\pgfmathprintnumber{#2}
    \else\ifnum#2<0
        -\pgfmathprintnumber{-#2}
    \fi\fi
    \right)
    \left(
    \ifnum#3=1
        x
    \else
        \pgfmathprintnumber{#3}x
    \fi
    \ifnum#4>0
        +\pgfmathprintnumber{#4}
    \else\ifnum#4<0
        -\pgfmathprintnumber{-#4}
    \fi\fi
    \right)
    = 
    \pgfmathparse{#1*#3} \pgfmathprintnumber{\pgfmathresult}x^2
    \pgfmathparse{#1*#4 + #2*#3}
    \ifnum\pgfmathresult>0
        +\pgfmathprintnumber{\pgfmathresult}x
    \else
        \pgfmathprintnumber{\pgfmathresult}x
    \fi
    \pgfmathparse{#2*#4}
    \ifnum\pgfmathresult>0
        +\pgfmathprintnumber{\pgfmathresult}
    \else
        \pgfmathprintnumber{\pgfmathresult}
    \fi
    \]
}



\newcommand{\expandbinomialsequation}[4]{%
    % Compute intermediate terms (A, B, C)
    \pgfmathparse{int(#1*#3)} \let\A\pgfmathresult  % A = a*c
    \pgfmathparse{int(#1*#4 + #2*#3)} \let\B\pgfmathresult  % B = a*d + b*c
    \pgfmathparse{int(#2*#4)} \let\C\pgfmathresult  % C = b*d

    % Compute absolute values and cast to integer
    \pgfmathparse{int(abs(\A))} \let\absA\pgfmathresult
    \pgfmathparse{int(abs(\B))} \let\absB\pgfmathresult
    \pgfmathparse{int(abs(\C))} \let\absC\pgfmathresult

    % Handle the signs for A, B, C
    \ifnum\A>0 \def\signA{} \else \def\signA{-} \fi
    \ifnum\B>0 \def\signB{+} \else \def\signB{-} \fi
    \ifnum\C>0 \def\signC{+} \else \def\signC{-} \fi

    % Print the quadratic expression in math mode with "= 0"
    \[
    \signA
    \ifnum\A=1
        x^2
    \else
        \pgfmathprintnumber{\A}x^2
    \fi
    \ifnum\B=0
    \else
        \signB
        \ifnum\absB=1
            x
        \else
            \pgfmathprintnumber{\absB}x
        \fi
    \fi
    \ifnum\C=0
    \else
        \signC\pgfmathprintnumber{\absC}
    \fi
    = 0
    \]   
}

\begin{document}
\section{ $Ax^2 + Bx +C$ where the y-intercept is positive In each case below solve for the roots using the factors method and sketch the graphs.} 

\begin{questions}
\LARGE

\question   $A+C=B$ When the ends  sum to the centre. 

\setlength{\columnsep}{20pt}
\begin{multicols}{2}
\begin{parts}
% Answer all \numparts   parts.
\part \expandbinomials{1}{1}{2}{3}  \vspace{45mm}
\part \expandbinomials{1}{1}{2}{5}  \vspace{45mm}
\part \expandbinomials{1}{1}{2}{7}  \vspace{45mm}
\part \expandbinomials{1}{1}{2}{11} \vspace{45mm}
\part \expandbinomials{1}{1}{2}{13} \vspace{45mm}
\part \expandbinomials{1}{1}{3}{5} 
\end{multicols}
\end{parts}
\vspace{45mm}

\question  Look for common patterns in the questions above. Where do A and C end up in your roots? What factor is common to all. What is the non-trivial root?




\newpage


\question   $A+C=-B$  Negative of outer sum is middle. What sign are both factors set to?
\setlength{\columnsep}{20pt}
\begin{multicols}{2}
\begin{parts}
\part \expandbinomials{1}{-1}{3}{-7}  \vspace{45mm}
\part \expandbinomials{1}{-1}{3}{-11} \vspace{45mm}
\part \expandbinomials{1}{-1}{3}{-13} \vspace{45mm}
\part \expandbinomials{1}{-1}{5}{-7} \vspace{45mm}
\part \expandbinomials{1}{-1}{5}{-11} \vspace{45mm}
\part \expandbinomials{1}{-1}{5}{-13} \vspace{45mm}
% \part \expandbinomials{1}{1}{7}{11} 
% \part \expandbinomials{1}{1}{7}{13} 
% \part \expandbinomials{1}{1}{11}{13}
\end{parts}
\end{multicols}
\question  Look for common patterns in the questions above. Where do A and C end up in your roots? What factor is common to all. What is the non-trivial root?

% \makebox[3cm]{\hrulefill}
\newpage

\question Positive Central Difference Cases.  
\setlength{\columnsep}{20pt}
\begin{multicols}{2}
\begin{parts}
\part \expandbinomials{1}{1}{8}{-5}  \vspace{22mm}
\part \expandbinomials{1}{-1}{3}{11} \vspace{22mm}
\part \expandbinomials{1}{-1}{8}{15} \vspace{22mm}
\part \expandbinomials{1}{1}{9}{-7} \vspace{22mm}
% \part \expandbinomials{1}{1}{11}{-5} \vspace{22mm}
% \part \expandbinomials{1}{-1}{4}{9} \vspace{22mm}
\end{parts}
\end{multicols}

\question Negative Central Difference Cases. 
\setlength{\columnsep}{20pt}
\begin{multicols}{2}
\begin{parts}

\part \expandbinomials{1}{1}{4}{-13} \vspace{22mm}
\part \expandbinomials{1}{-1}{11}{7} \vspace{22mm}
\part \expandbinomials{1}{1}{7}{-9} \vspace{22mm}
\part \expandbinomials{1}{1}{11}{-13}\vspace{22mm}
% \expandbinomials{1}{1}{6}{-11}\vspace{22mm}
% \expandbinomials{1}{1}{5}{-7}\vspace{22mm}
% \expandbinomials{1}{-1}{7}{2}\vspace{22mm}
% \expandbinomials{1}{-1}{11}{3}\vspace{22mm}
\end{parts}
\end{multicols}
\question How is the sign of the dominant cross product related to the sign of the $x$ term in both questions above?
\newpage

% \question Monic or $x^2$  quadratic negative constant trivial case. 
% \setlength{\columnsep}{20pt}
% \begin{multicols}{3}
% \begin{parts}
% \part \expandbinomials{1}{1}{1}{-7}  \vspace{45mm}
% \part \expandbinomials{1}{1}{1}{-6} \vspace{45mm}
% \part \expandbinomials{1}{1}{1}{5} \vspace{45mm}
% \part \expandbinomials{1}{1}{1}{-7} \vspace{45mm}
% \part \expandbinomials{1}{1}{1}{3} \vspace{45mm}
% \part \expandbinomials{1}{1}{1}{13} \vspace{45mm}
% \part \expandbinomials{1}{-1}{1}{11} \vspace{45mm}
% \part \expandbinomials{1}{-1}{1}{13} \vspace{45mm}
% \part \expandbinomials{1}{-1}{1}{13}\vspace{45mm}
% \end{parts}
% \end{multicols}

% \newpage



\question Trivial  ($x^2 +Bx +C$ )   case. 
\setlength{\columnsep}{20pt}
\begin{multicols}{3}
\begin{parts}
\part \expandbinomials{1}{1}{1}{17}  \vspace{25mm}
\part \expandbinomials{1}{1}{1}{19} \vspace{25mm}
 \part \expandbinomials{1}{1}{1}{20} \vspace{25mm}
% \part \expandbinomials{1}{1}{1}{29} 

\end{parts}
\end{multicols}
\vspace{25mm}



\question Trivial  ($Ax^2 +Bx +1$ )   case. 
\setlength{\columnsep}{20pt}
\begin{multicols}{3}
\begin{parts}
\part \expandbinomials{5}{1}{1}{1}  \vspace{25mm}
\part \expandbinomials{16} {1}{1}{1}\vspace{25mm}
\part \expandbinomials{40}{1}{1}{1} \vspace{25mm}
% \part \expandbinomials{59} {1}{1}{1}

\end{parts}
\end{multicols}
\vspace{25mm}

\question Trivial  ($Ax^2 /pm Bx -1$ )   case.
\setlength{\columnsep}{20pt}
\begin{multicols}{3}
\begin{parts}
\part \expandbinomials{5}{1}{1}{-1}  \vspace{35mm}
\part \expandbinomials{11}{-1}{1}{1} \vspace{35mm}
\part \expandbinomials{42}{1}{1}{-1}  \vspace{35mm}
% \part \expandbinomials{8}{1}{1}{-1} \vspace{35mm}
\end{parts}
\end{multicols}
\newpage

















\question Simplest  $x^2$ with negative constant case in where improper factors are needed? 
\setlength{\columnsep}{20pt}
\begin{multicols}{3}
\begin{parts}
\part \expandbinomials{1}{1}{1}{-7}  \vspace{45mm}
\part \expandbinomials{1}{1}{1}{-6} \vspace{45mm}
\part \expandbinomials{1}{1}{1}{-30} \vspace{45mm}
\part \expandbinomials{1}{1}{1}{-37} \vspace{45mm}

\end{parts}
\end{multicols}


\question  Simplest case for $-1$ constant. 
\setlength{\columnsep}{20pt}
\begin{multicols}{3}
\begin{parts}
\part \expandbinomials{5}{1}{1}{-1}  \vspace{45mm}
\part \expandbinomials{11}{-1}{1}{1} \vspace{45mm}
\part \expandbinomials{42}{1}{1}{-1}  \vspace{45mm}
\part \expandbinomials{8}{1}{1}{-1} \vspace{45mm}

\end{parts}
\end{multicols}
\newpage



\end{questions}

\end{document}
