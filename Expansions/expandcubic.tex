
\documentclass[12pt, a4paper, addpoints]{exam}

\title{Synthetic Division}


\pagestyle{empty} % Suppress page numbers
\usepackage[top=5mm, bottom=20mm, left=11mm, right=11mm]{geometry}
\usepackage{amsmath} 
\usepackage{multicol} 
\usepackage{tabularx}
\usepackage{pgfmath}
\usepackage{xcolor} 
\pagestyle{empty}
% \author{Mr Downes}
\newcommand{\ts}{\vspace{16mm}}
\newcommand{\ms}{\vspace{20mm}}
\newcommand{\bs}{\vspace{30mm}}
\newcommand{\ls}{\vspace{30mm}}
\newcommand{\hs}{\vspace{30mm}}
\usepackage{array}     % For centering text vertically
\usepackage{graphicx}  % For rotating text

\date{}



% Define the cubic expansion
\newcommand{\expandquadratic}[5]{
    \pgfmathsetmacro{\termone}{int(#1*#4)} % Coefficient of x^3
    \pgfmathsetmacro{\termtwo}{int(#1*#5 + #2*#4)} % Coefficient of x^2
    \pgfmathsetmacro{\termthree}{int(#2*#5 + #3*#4)} % Coefficient of x
    \pgfmathsetmacro{\termfour}{int(#3*#5)} % Constant term
    
    \ifnum\termone=0\else
        \ifnum\termone=1 x^3\else
        \ifnum\termone=-1 -x^3\else \termone x^3\fi\fi
    \fi
    \ifnum\termtwo=0\else
        \ifnum\termtwo>0 
            \ifnum\termtwo=1 + x^2\else + \termtwo x^2\fi
        \else
            \ifnum\termtwo=-1 - x^2\else \termtwo x^2\fi
        \fi
    \fi
    \ifnum\termthree=0\else
        \ifnum\termthree>0 
            \ifnum\termthree=1 + x\else + \termthree x\fi
        \else
            \ifnum\termthree=-1 - x\else \termthree x\fi
        \fi
    \fi
    \ifnum\termfour=0\else
        \ifnum\termfour>0 + \termfour\else \termfour\fi
    \fi
}

% Define synthetic division
\newcommand{\syntheticdiv}[5]{
    \dfrac{
        % Call the expandquadratic to generate the cubic expression
        \expandquadratic{#1}{#2}{#3}{#4}{#5}
    }{
        % Denominator (linear factor)
        \ifnum#4=1
            x \pm #5
        \else
            #4x \pm #5
        \fi
    }
}


\begin{document}
% Adjust vertical spacing before and after the title

\maketitle
\vspace{-28mm}
% \section{Calculations using powers  $a^p$  and $a^q$ }
 % \huge

\large

\begin{questions}



% \question Recall and  understanding these indices and logarithms identities from page 21.
% \Large
% % \begin{multicols}{2}
% \begin{parts}
% \part \text{Product of Powers — Add the exponents} \hfill $a^p a^q = a^{p+q}$

% \part \text{Quotient of Powers — Subtract  the exponents} \hfill $\dfrac{a^p}{a^q} = a^{p-q}$
% \part \text{Power of a Power — Multiply the exponents} \hfill $(a^p)^q = a^{pq}$

% \part \text{Zero Exponent — Any non-zero base raised to zero equals 1:} \hfill $a^0 = 1$

% \part \text{Negative Exponent — Recipricate  and change the sign} \hfill $a^{-p} = \dfrac{1}{a^p}$


% % \part \text{Power of a Quotient} \hfill $\left( \dfrac{a}{b} \right)^p = \frac{a^p}{b^p}$

% \part \text{Power of a Product — Distribute the exponent} \hfill $(ab)^p = a^p b^p$


% \end{parts}
% % \end{multicols}

\large

\question Apply the Product of Powers rule — Add the exponents to simplify.

\setlength{\columnsep}{20pt}
\begin{multicols}{2}
\begin{parts}
\part \syntheticdiv{2}{3}{4}{1}{5}
\part \expandquadratic{1}{6}{-17}{2}{5}
\part \expandquadratic{1}{-10}{-3}{1}{7}
\end{parts}
\end{multicols}
\ts





\end{questions}


\end{document}
