\documentclass[12pt, a4paper, addpoints]{exam}
\usepackage[top=15mm, bottom=20mm, left=5mm, right=5mm]{geometry}
\usepackage{pgfmath} % For random numbers
\usepackage{multicol} % For multi-column layout

% Randomize seed using the current time
\pgfmathsetseed{\number\time}

% Define commands for different expansions

% Monicpositivefactor lead, Nonmonicpositivefactor trail
\newcommand{\monicleadnonmonictrail}{%
    \pgfmathtruncatemacro{\a}{random(2,5)} % Random non-monic coefficient for a
    \pgfmathtruncatemacro{\b}{random(1,5)} % Random constant for monic
    \pgfmathtruncatemacro{\c}{random(1,5)} % Random constant for non-monic
    \edef\result{(x + \b)(\a x + \c)}%
    \result
}

% Nonmonicpositivefactor lead, Monicnegativefactor trail
\newcommand{\nonmonicleadmonictrail}{%
    \pgfmathtruncatemacro{\a}{random(2,5)} % Random non-monic coefficient for a
    \pgfmathtruncatemacro{\b}{random(1,5)} % Random constant for non-monic
    \pgfmathtruncatemacro{\c}{random(1,5)} % Random constant for monic
    \edef\result{(\a x + \b)(x - \c)}%
    \result
}

% Monicpositivefactor lead and trail (perfect square)
\newcommand{\monicpositiveleadtrail}{%
    \pgfmathtruncatemacro{\b}{random(1,5)} % Random constant
    \edef\result{(x + \b)(x + \b)}%
    \result
}

% Monicnegativefactor lead and trail (perfect square)
\newcommand{\monicnegativeleadtrail}{%
    \pgfmathtruncatemacro{\b}{random(1,5)} % Random constant
    \edef\result{(x - \b)(x - \b)}%
    \result
}

% Nonmonicpositivefactor lead and trail
\newcommand{\nonmonicpositiveleadtrail}{%
    \pgfmathtruncatemacro{\a}{random(2,5)} % Random non-monic coefficient
    \pgfmathtruncatemacro{\b}{random(1,5)} % Random constant
    \edef\result{(\a x + \b)(\a x + \b)}%
    \result
}

% Nonmonicnegativefactor lead and trail
\newcommand{\nonmonicnegativeleadtrail}{%
    \pgfmathtruncatemacro{\a}{random(2,5)} % Random non-monic coefficient
    \pgfmathtruncatemacro{\b}{random(1,5)} % Random constant
    \edef\result{(\a x - \b)(\a x - \b)}%
    \result
}

\begin{document}

\section*{Expand and Break Down Using FOIL}
\quad Student's Name: \underline{\hspace{5cm}}

\begin{questions}
\LARGE

% Question 1: Monic Positive Lead, Non-Monic Positive Trail
\question
Expand the following using FOIL and break down the steps (First, Outer, Inner, Last):
\setlength{\columnsep}{20pt}
\begin{multicols}{2}
\begin{parts}
    \part \( \monicleadnonmonictrail \)
    \ps
    % First: x * ax = ax^2
    % Outer: x * c = xc
    % Inner: b * ax = abx
    % Last: b * c = bc
\end{parts}
\end{multicols}

% Question 2: Non-Monic Positive Lead, Monic Negative Trail
\question Non-Monic Positive Lead, Monic Negative Trail
Expand the following using FOIL and break down the steps (First, Outer, Inner, Last):
\setlength{\columnsep}{20pt}
\begin{multicols}{2}
\begin{parts}
    \part \( \nonmonicleadmonictrail \)
    \ps
    % First: ax * x = ax^2
    % Outer: ax * -c = -acx
    % Inner: b * x = bx
    % Last: b * -c = -bc
\end{parts}
\end{multicols}

% Question 3: Monic Positive Perfect Square
\question Monic Positive Perfect Square
Expand the following using FOIL and break down the steps:
\setlength{\columnsep}{20pt}
\begin{multicols}{2}
\begin{parts}
    \part \( \monicpositiveleadtrail \)
    \ps
    % First: x * x = x^2
    % Outer: x * b = xb
    % Inner: b * x = bx
    % Last: b * b = b^2
\end{parts}
\end{multicols}

% Question 4: Non-Monic Negative Perfect Square
\question
Expand the following using FOIL and break down the steps:
\setlength{\columnsep}{20pt}
\begin{multicols}{2}
\begin{parts}
    \part \( \nonmonicnegativeleadtrail \)
    \ps
    % First: ax * ax = a^2x^2
    % Outer: ax * -b = -abx
    % Inner: -b * ax = -abx
    % Last: -b * -b = b^2
\end{parts}
\end{multicols}

\end{questions}

\end{document}
