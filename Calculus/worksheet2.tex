\documentclass[12pt, a4paper, addpoints]{exam}

\title{The Derivative of a Function and Vertex Form of a Quadratic}

\usepackage{amsmath}
\usepackage{array}
\usepackage{booktabs}
\pagestyle{empty} % Suppress page numbers
\usepackage[top=5mm, bottom=20mm, left=11mm, right=11mm]{geometry}
\usepackage{amsmath} % For mathematical symbols
\usepackage{multicol} % For multi-column layout
\usepackage{pgfplots} % For graphing
\usepackage{tabularx}
\usepackage{pgfmath} % For math parsing and calculations
\usepackage{xcolor} % For color customization
\pagestyle{empty}

% Custom spacing
\newcommand{\ts}{\vspace{6 mm}}
\newcommand{\ms}{\vspace{11 mm}}
\newcommand{\bs}{\vspace{14mm}}
\newcommand{\ls}{\vspace{33 mm}}
\newcommand{\hs}{\vspace{44 mm}}

% A command to draw a quadratic function in vertex form y = a(x - h)^2 + k with limits -6 to 6 on x-axis
\newcommand{\plotquadratic}[4]{%
\begin{tikzpicture}
    \begin{axis}[
        axis lines = middle,
        xlabel = $x$, ylabel = $y$,
        grid = both,
        width=8cm, height=6cm,
        xmin=-6, xmax=6, ymin=-10, ymax=10,
        samples=100
    ]
    \addplot [domain=-6:6, thick, color=blue] {#1*(x - #2)^2 + #3};
    \node at (axis cs:#2,#3)[pin=above:{Turning Point: (#2,#3)}] {};
    \end{axis}
\end{tikzpicture}
}

\date{}
\begin{document}
% Adjust vertical spacing before and after the title

\maketitle
\vspace{-28mm}

\begin{questions}

\question Learning intentions: Recall, understand, and apply the following:

\begin{parts}
\part Find the turning point of a curve
\part Find the second derivative
\part Use the first derivative to find turning points
\part Use the second derivative to find the nature of turning points
\end{parts}

\large

% Vertex form and turning points
\question For each of the following quadratic functions, plot the graph and calculate the turning point. Use the vertex form $y = a(x-h)^2 + k$. Also, expand the quadratic into standard form $y = ax^2 + bx + c$ and simplify.

\begin{multicols}{2}
\begin{parts}
   \scalebox{0.8}
    \part $y = (x - 2)^2 + 3$
    \plotquadratic{1}{2}{3}{}
    % Standard form: y = x^2 - 4x + 7
    Standard form: $y = x^2 - 4x + 7$ \ms
    
 \scalebox{0.8}
    \part $y = -2(x + 1)^2 + 4$
    \plotquadratic{-2}{-1}{4}{}
    % Standard form: y = -2x^2 - 4x + 2
    Standard form: $y = -2x^2 - 4x + 2$ \ms
    
    % Third quadratic
    \part $y = 3(x - 1)^2 - 2$
    \plotquadratic{3}{1}{-2}{}
    % Standard form: y = 3x^2 - 6x + 1
    Standard form: $y = 3x^2 - 6x + 1$ \ms
    
    % Fourth quadratic
    \part $y = -x^2 + 5$
    \plotquadratic{-1}{0}{5}{}
    % Standard form: y = -x^2 + 5 (Already in standard form)
    Standard form: $y = -x^2 + 5$ \ms
\end{parts}
\end{multicols}
\ts

\question Find the turning point of each quadratic function and sketch including vertex point.

\begin{multicols}{3}
\begin{parts}\Large
    \part $y = x^2 - 4x + 7$ \ms
    \part $y = -2x^2 - 4x + 2$ \ms
    \part $y = 3x^2 - 6x + 1$ \ms
    \part $y = x^2 +8x+ 5$ \ms
    \part $y = 4x^2 + 12x - 6$ \ms
    \part $y = -3x^2 + 24x + 2$ \ms
\end{parts}
\end{multicols}
\ts

\question Find the turning points (tps) of each cubic function and sketch the graph, including the turning points.


\begin{parts}\Large
    \part $y = x^3 - 3x^2$ \\ (tps: $(0, 0)$, $(2, 4)$) \bs
    \part $y = x^3 - 6x^2 + 9x - 10$ \\ (tps: $(1, -6)$, $(3, -10)$) \bs
    \part $y = x^3 - 3x^2 - 9x + 6$ \\ (tps: $(-1, 11)$, $(3, -24)$) \bs
    \part $y = 2x^3 - 3x^2 - 12x + 7$ \\ (tps: $(-1, 20)$, $(2, -25)$) \bs
    \part $y = 2x^3 - 9x^2 + 12x$ \\ (tps: $(0, 0)$, $(3, 0)$) \bs
    \part $y = x^3 - 6x^2 + 9x$ \\ (tps: $(0, 0)$, $(3, 0)$) \bs
\end{parts}


\question HW 10.4 page 253 -256
\end{questions}

\end{document}











\end{questions}

\end{document}
