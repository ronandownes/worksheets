\documentclass[12pt, a4paper, addpoints]{exam}
\usepackage[ddmmyy, useregional]{datetime}
\title{Factor, Solve and Sketch  $x^2+bx+c=0$ \dayofweekname{\day}{\month}{\year} \today}

\usepackage{amsmath}
\usepackage{array}
\usepackage{booktabs}
\pagestyle{empty} % Suppress page numbers
\usepackage[top=5mm, bottom=20mm, left=11mm, right=11mm]{geometry}
\usepackage{amsmath} % For mathematical symbols
\usepackage{multicol} % For multi-column layout
\usepackage{pgfplots} % For graphing
\usepackage{tabularx}
\usepackage{pgfmath} % For math parsing and calculations
\usepackage{xcolor} % For color customization
\pagestyle{empty}

% Custom spacing
\newcommand{\ts}{\vspace{22 mm}}
\newcommand{\ms}{\vspace{33 mm}}
\newcommand{\bs}{\vspace{44mm}}
\newcommand{\ls}{\vspace{44 mm}}
\newcommand{\hs}{\vspace{44 mm}}

\newcommand{\monicquad}[2]{x^2 - (#1 + #2)x + #1#2}
\newcommand{\cubicwithtps}[3]{
    \pgfmathtruncatemacro{\a}{3*#1}  % x^2 term (after integration and multiplying by 6)
    \pgfmathtruncatemacro{\b}{6*#2}  % x term (after integration and multiplying by 6)

    $y = 2x^3 
    \ifnum\a>0 + \a x^2 \else \a x^2 \fi 
    \ifnum\b>0 + \b x \else \b x \fi 
    \ifnum#3>0 + #3 \else #3 \fi$
}
\usepackage{mdframed}
\date{}
\begin{document}
% Adjust vertical spacing before and after the title
\maketitle

\vspace{-18mm}
\noindent 
In todays class we connect  numerical, graphical and algebraic concepts to improve your fluency, enrich  your knowledge and deepen your understanding of finding and applying solutions to quadratic equations
\begin{questions}


% \section*{Circle-Test:    \quad  Students Name_ \underline{\hspace{8cm}}}

\begin{mdframed}[backgroundcolor=gray!10] % Light grey background
% \textbf{Learning Intentions and keywords:}
\scriptsize
\setlength{\columnsep}{2pt}
\begin{multicols}{4}
\begin{itemize}
\item  Expanding and factoring are inverse operations
\item Vieta's formulas: $x^2 + bx + c = 0$, the \textbf{sum of the roots} is $-b$ and the \textbf{product of the roots} is $c$.
\item Special cases when a   root is \(+ 1  \textbf{ or } -1\)
\item Otherwise proper factors with  sum or difference linear coifficient
\item   $\mathbb{N}$ with addition and subtraction
is simpler than  $\mathbb{Z}$ approach
\item \(  x^2 + bx + c \)  is in descending power in x
\item Connect special cases with multiples of 11
    \item Same-sign positive factors
    \item Mixed-sign positive dominated factors
   \item Mixed-sign negative dominated factors
    \item Zero-product rule
    \item Roots of an equation
    \item Sketching quadratice
    \item Parabola
    \item Expanded form, factored form and vertex form
\end{itemize}
\end{multicols}



\end{mdframed}





\question Special  case  of  positive factors where one factor  is \((x+1) \).


\begin{parts}\Large
\begin{multicols}{3}
    \part $ x^2 +8x  + 7$ \ms
    \part $ x^2 +13x + 12$ \ms
    \part $x^2 + 18+ 19x  $ \ms
\end{multicols} \ms
% \part \small What  properties are common in this case? 
\end{parts}

\ts

\question Special  case  of negative factors where one factor is \((x-1) \). 

\begin{multicols}{3}
\begin{parts}\Large
    \part $x^2 - 4x + 3$ \ms
    \part $x^2 - 11x + 10$ \ms
    \part $x^2  + 10- 11x $ \ms
\end{parts}
\end{multicols}
\ts

% \resizebox{\textwidth}{!}{Your}

\question Mixed-sign  with  \((x+1) \) special case. 
\begin{multicols}{3}
\begin{parts}\Large
    \part $x^2 -4x-5$ \ms
   \part $x^2 -8x-9$ \ms
    \part $x^2 -23x-24$ \ms
\end{parts}
\end{multicols}
\ts

\question Special  case  of mixed-sign  factors where one factor is \((x-1) \). 
\begin{multicols}{3}
\begin{parts}\Large
    \part $x^2 +7x+10$ \ms
   \part $x^2 +15x+26$ \ms
    \part $x^2 -13x+30$ \ms
\end{parts}
\end{multicols}
\ts




\question Solve and sketch 

\begin{multicols}{3}
\begin{parts}\Large
    \part $x^2 +5x-14$ \ms
   \part $x^2 +9x-33$ \ms
    \part $yx^2 -2x-35$ \ms

\end{parts}
\end{multicols}
\ts




\question Find the turning points and indicate on a sketch the max and min points

\begin{multicols}{2}
\begin{parts}\Large
\part \cubicwithtps{6}{5}{-11}  \ts
\part \cubicwithtps{7}{10}{-19}  \ts
\part \cubicwithtps{3}{-4}{2}   \ts 
\part \cubicwithtps{-5}{-14}{-11}  \ts
\part \cubicwithtps{4}{-21}{8}   \ts 
\part \cubicwithtps{7}{10}{-19}  \ts
\end{parts}
\end{multicols}
\bs


\end{questions}

\end{document}
